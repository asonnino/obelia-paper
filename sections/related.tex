\section{Discussion and Related Work} \label{sec:related}
\sysname conceptually aligns with IOTA's open writing access~\cite{cullen2021access,muller2022tangle,iota-wiki} where all validators have writing privileges, and a ``leader'' validates blocks.
%
\sysname however operates fundamentally differently, as it is designed as an add-on for existing quorum-based blockchains. It draws inspiration from DagRider's weak links~\cite{dag-rider}, which include older blocks not required for leader selection (although for different reasons than \sysname), and Narwhal's vertex-creation rule~\cite{narwhal}, which ensures vertex availability before inclusion in the DAG.
%
Future work include the analysis of \sysname where auxiliary validators operated under the sleepy model~\cite{pass2017sleepy} to explore potential improvements in censorship resistance; whether auxiliary validators can enhance the protocol's safety for clients who trade latency, as in OFlex~\cite{malkhi2019flexible, oflex}; and whether they can aid in fork recovery when more than $f$ Byzantine core validators are present. Lastly, we leave as future work the incentive analysis and its impact on the relationship between $n_c$ and $n_a$ ($n_c < n_a$, $n_c > n_a$, or $n_c \approx n_a$).