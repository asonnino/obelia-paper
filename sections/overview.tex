\section{System Overview} \label{sec:overview}

We present the settings in which \sysname operates.

\subsection{Validators selection} \label{sec:validators}
\sysname introduces the distinction between \emph{core validators} and \emph{auxiliary validators}. Core validators are the validators that \emph{continuously} operate the consensus protocol and process transactions. In contrast, auxiliary validators participate sporadically while maintaining a copy of the DAG generated by core validators.
Both core and auxiliary validators are selected using a sybil-resistant mechanism~\cite{douceur2002sybil}, typically based on proof-of-stake~\cite{kucci2021proof}. Core validators are chosen similarly to existing quorum-based blockchains, consisting of roughly the 100 entities with the highest stake or those meeting specific criteria, such as owning a minimum percentage of the total stake~\cite{sui}.  Auxiliary validators include all other stakeholding entities not in the core group, typically numbering in the several hundreds~\cite{sui-scan}, significantly surpassing the number of core validators.  In practice, we expect current full nodes to operate as auxiliary validators.

\subsection{System and threat model} \label{sec:model}
\sysname assumes a computationally bounded adversary, ensuring the security of cryptographic properties such as hash functions and digital signatures. It operates as a message-passing system where core and auxiliary validators collectively hold  $n = n_c + n_a$ units of stake~\cite{saad2020comparative}, with $n_c$ held by core validators and $n_a$ held by auxiliary validators. Each unit of stake represents one ``identity''~\cite{douceur2002sybil}, while each unit held by a core validator signifies one ``unit of voting power'' in the consensus system~\cite{sui-code,sui}. This model aligns with deployed quorum-based blockchains, where core validators possess the majority of total stake ($n_c \gg n_a$)~\cite{sui,aptos,solana}. \sysname makes the following assumptions for core and auxiliary validators:

\para{Core validators}
\sysname works with existing DAG-based consensus protocols, inheriting their assumptions. Specifically, it requires that $n_c \geq 3f + 1$, where $f$ is the maximum number of \emph{Byzantine} core validators that may deviate from the protocol. The remaining stake is held by \emph{honest} core validators who adhere to the protocol. There are no additional assumptions about the network model, core validators operate in the same setting as the underlying DAG-based consensus protocol. Note that most deployed DAG-based consensus protocols are partially synchronous~\cite{dwork1988consensus}, while some blockchains consider asynchronous protocols~\cite{sui-code}. Under these assumptions, \Cref{sec:security} demonstrates that a DAG-based protocol enhanced with \sysname is \emph{safe}, meaning no two correct validators can commit conflicting transactions.

\para{Auxiliary validators}
For auxiliary validators, \sysname adopts a relaxed model due to their lower stake and reduced incentives for resource dedication and reliability. It assumes that at least $t_a \leq n_a$ units of stake are consistently held by honest and active auxiliary validators, regardless of the total number of auxiliary validators. The parameter $t_a$ can be adjusted to balance system liveness (see \Cref{sec:security}) against the minimum participation of auxiliary validators. Auxiliary validators do not communicate with one another and only occasionally communicate with core validators over an asynchronous network. \Cref{sec:security} shows that, under these assumptions, a DAG-based protocol enhanced with \sysname is \emph{live}, ensuring that honest validators eventually commit transactions. Importantly, if the assumptions concerning auxiliary validators fail, safety remains guaranteed.

\subsection{Design goals and challenges} \label{sec:goals}

Beyond ensuring safety and liveness within the same network model as the underlying consensus protocol, \sysname achieves several design goals (discussed in \Cref{sec:design}):
\textbf{Increased participation (G1):} It allows all entities holding stake to author blocks in the consensus protocol, rather than limiting participation to the top 100 validators.
\textbf{Incentivized synchronizer helpers (G2):} \sysname leverages auxiliary validators to assist slow or recovering core validators in catching up to the latest state. This approach incentivizes auxiliary validators to function as full nodes, storing and providing the DAG state to core validators to facilitate synchronization.
\textbf{Generic design (G3):} The design of \sysname is directly applicable to a wide range of structured DAG-based consensus protocols.

\sysname also has performance goals that we demonstrate empirically in \Cref{sec:evaluation}:
\textbf{Negligible overhead (G4):} \sysname introduces minimal overhead, allowing the system to progress at the same speed as the underlying consensus protocol.
\textbf{Scalability (G5):} \sysname scales effectively with the number of auxiliary validators.
\textbf{Fault tolerance (G6):} \sysname maintains robust performance, remaining visibly unaffected by the presence of crashed auxiliary validators.

To achieve these goals, \sysname overcomes several challenges:
\textbf{(Challenge 1):} \sysname cannot implement an all-to-all communication design due to the impractical number of auxiliary validators.
\textbf{(Challenge 2):} \sysname cannot expect the classic BFT assumption to hold for these entities as they have less stake and thus less incentive to be reliable and prone to remain offline for long periods of time.
\textbf{(Challenge 3):} Auxiliary validators must participate in the consensus without causing delays, as this would undermine the key advantage of quorum-based systems. They thus cannot take actions that impact the critical path.
